\chapter*{Úvod}
\addcontentsline{toc}{chapter}{Úvod}

Prostředí, v němž se lidé každý den pohybují, je komplexní. Abychom se v něm
zorientovali, potřebujeme často v našem okolí zrakem najít nějaký objekt. Tato
činnost se jmenuje zrakové vyhledávání. O zrakovém vyhledávání existují mnohé
výzkumy. Jeden z aspektů vizuálního a obecně smyslového vnímání, který se
zkoumá, je vztah mezi konkrétními stimuly a jim odpovídajícími vjemy. Tuto
závislost zkoumá psychologická disciplína jménem psychofyzika.

Reálný svět je ale obtížně popsatelný. Proto se k experimentům často
používají zjednodušené modely, které se v některých aspektech blíží reálným
situacím, ale jsou snadno formalizovatelné. Jedna z výhod formálního modelu je
možnost zkonstruovat model ideálního pozorovatele. Ten je důležitý pro srovnání
úspěšnosti, které dosahuje lidský pozorovatel.

Jedním z úkolů, který se řeší, je vyhledávání konkrétního cíle v poli s vizuálním šumem.
Najemnik a Geisler \citeyearpar{Najemnik05} v nedávné době představili model
Ideálního Bayesovského pozorovatele\index{Ideální Bayesovský pozorovatel}, který modeluje chování lidského
pozorovatele s překvapivou přesností. Algoritmus, pomocí kterého Ideální
Bayesovský pozorovatel vybíral lokace, které zkoumat, byl ale kubický v počtu
možných lokací cíle. Ve svém pozdějším článku \citep{Najemnik09} 
představili pozorovatele ELM\index{ELM pozorovatel}, který Ideálního Bayesovského pozorovatele dobře
aproximuje, a navíc pracuje v kvadratickém čase.

Existence kvadratického algoritmu pro hodnocení kvality otevírá novou možnost:
Vzhledem k typickému počtu možných lokací cíle v dané úloze probíhá výpočet jednoho kroku
tak rychle, že je z pohledu uživatele okamžitý. Můžeme tedy v reálném
čase hodnotit, jak dobře či špatně vůči ELM modelu vybírá lidský pozorovatel
lokace, kam se podívá.

Zajímavá otázka je, jakým způsobem se lidský pozorovatel dokáže nejlépe naučit
řešit zmíněnou úlohu zrakového vyhledávání (to znamená dosáhnout co nejnižší
střední hodnoty počtu pohledů potřebných k nalezení cíle).

Pokud se člověk například bude chtít naučit běhat, nabízí se dva možné přístupy.
Buď může pouze sám zkoušet běhat, nebo si může nechat radit od trenéra. V prvním
případě má po běhu pouze jedinou zpětnou vazbu: ví, jak se cítí, za jaký čas
danou vzdálenost uběhl a podobně. Neví však, jakou měrou za to může jeho běh a 
jakou náhoda. Trenér naproti tomu může podat například zpětnou vazbu typu \uv{Styl běhu byl nevhodný,
a přesto, že se tentokrát nic nestalo, opakováním tohoto postupu může dojít ke zranění}.

Analogicky k tomuto příkladu máme i v naší úloze dva možné přístupy.
Učení může probíhat pouze opakovaným řešením této úlohy. V takovém případě bude
jedinou zpětnou vazbou pozorovatele informace o tom, jak rychle úlohu vyřešil. Druhou možností je,
že dostane při každém pohledu informaci, jak dobře či špatně si vybral lokaci podle
hodnotící funkce ELM pozorovatele.

V této práci vytvoříme aplikaci, která úlohu zrakového vyhledávání gamifikuje. S její pomocí provedeme experiment, jehož cílem bude zjistit, zda má dodatečná informace z modelu ELM pozorovatele pozitivní vliv na to, jak rychle se účastník po daném počtu pokusů naučí cíl hledat.




\section*{Struktura práce}
\addcontentsline{toc}{section}{Struktura práce}

V první kapitole jsou představeny pojmy a teorie, které tato
práce využívá, a starší výsledky, z kterých vychází. V druhé kapitole jsou představeny cíle práce. V následující
kapitole je popsána metodika experimentu. Poté jsou prezentovány vizualizace
naměřených dat, nad nimiž je následně provedena diskuse. 

K práci jsou též
přiloženy dvě přílohy. V první z nich jsou další grafy z naměřených dat, v
druhé se nachází dokumentace aplikace, prostřednictvím které byl prováděn
experiment a jejíž vznik byl součástí této práce.

