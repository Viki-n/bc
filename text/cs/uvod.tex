\chapter*{Úvod}
\addcontentsline{toc}{chapter}{Úvod}

Jedna z činností, kterou provádíme mnohokrát denně, je zrakové vyhledáváni.
Proto také existují mnohé výzkumy o tom, jak člověk zrakové vyhledávání
provádí. Jeden z oborů, který se o zrakové vyhledávání zajímá, je psychofyzika.

Jedním z úkolů, který se řeší, je vyhledávání konkrétního cíle v kruhovém poli.
Najemnik a Geisler \citeyearpar{Najemnik05} v nedávné době představili model
Ideálního Bayesovského pozorovatele, který modeluje chování lidského
pozorovatele s překvapivou přesností. Algoritmus, pomocí kterého Ideální
Bayesovský pozorovatel vybíral lokace, které zkoumat, byl ale kubický v počtu
možných lokací cíle. Ve svém pozdějším článku \citep{Najemnik09} 
představili pozorovatele ELM, který Ideálního Bayesovského pozorovatele dobře
aproximuje, a navíc pracuje v kvadratickém čase.

V této práci se podíváme na dva dosud neřešené problémy. Prvním z nich je, zda se lidští
pozorovatelé chovají stejně, když se musí vědomě rozhodovat, jaké na jaké místo
se ve kterém kroku podívají. Druhou, navazující otázkou je, zda se člověk naučí
cíl hledat lépe, když to bude sám zkoušet, nebo když při učení dostane navíc po každém pohledu
informaci o tom, jak dobře si vybral místo, na které se podíval, srovnáním s hodnotou, kterou
tomuto bodu přiřazuje hodnotící funkce ELM pozorovatele.

\section*{Struktura práce}
\addcontentsline{toc}{section}{Struktura práce}

V první kapitole jsou představeny pojmy, teorie a starší výsledky, které tato
práce využívá. V druhé kapitole jsou představeny cíle práce. V následující
kapitole je popsána metodika experimentu. Poté jsou prezentovány vizualizace
naměřených dat, nad nimiž je následně provedena diskuse. K práci jsou též
přiloženy dvě přílohy. V první z nich jsou další grafy z naměřených dat, v
druhé se nachází dokumentace aplikace, prostřednictvím které byl prováděn
experiment a jejíž vznik byl součástí této práce.

