\chapter{Cíle práce}

Cílem této práce je zjistit, zda se lidé dokáží naučit hledat konkrétní Gabor
patch v kruhovém růžovém šumu lépe, pokud jim při tréninku budeme po každé
fixaci dávat akusticky informaci o tom, jak dobrý nápad byl zvolit k fixaci
právě bod, který zvolili. Jejich schopnost najít cíl budeme hodnotit
průměrným počtem fixací potřebných k nalezení cíle. Tohoto cíle dosáhneme
následujícími kroky:

\begin{enumerate}

\item Vytvoříme aplikaci pro operační systém iOS, v níž bude uživatel hledat
Gabor patch v kruhovém poli a volitelně bude dostávat zpětnou vazbu, jako byla
popsána v předchozím odstavci.

\item Navrhneme experiment, během nějž budeme pomocí zmíněné aplikace nejprve
měřit schopnost účastníků hledat Gabor patch, poté se budou účastníci učit
hledat Gabor patch, a nakonec opět tuto schopnost změříme.

\item Najdeme několik účastníků, které náhodně rozdělíme do dvou skupin,
kontrolní a experimentální. Všechny účastníky necháme projít experimentem
navrženým v předchozím bodě. Účastníci v kontrolní skupině se budou učit hledat
Gabor patch pouze tím, že ho budou opakovaně hledat. Účastníci v experimentální
skupině budou navíc ve fázi učení dostávat zpětnou vazbu.

\item Výsledky experimentu zpracujeme běžnými statistickými metodami, aby\-chom
zjistili, zda se účastníci v experimentální skupině zlepšili během učení více,
než účastníci v kontrolní skupině, a pokud ano, tak zda je rozdíl mezi
skupinami statisticky významný.

\end{enumerate} 
