\chapter{Základní pojmy}

\section{Šum}

Zeptat se Fíďy, je to fakt formálně pink noise?




\section{Gabor patch}

Gabor filter (v českých textech někdy označovaný jako Gaborova vlnka) je
lineární filtr používaný ve zpracování obrazu, chceme-li detekovat signál
mající danou frekvenci a směr, který se vyskytuje kolem daného bodu.
Chceme-li detekovat signál ve vizuálním šumu, jenoduše spočítáme hodnotu filtru
pro každý bod obrazu a vynásobíme ho s hodnotou šumu. Poté spočítáme součet
všech takto získaných hodnot. Je-li součet blízko nuly, signál v daném místě
není přítomen, nebo je přítomen s jinými parametry. Vysoké hodnoty značí, že
signál pravděpodobně přítomen je, hluboce záporné značí, že signál je přítomen,
ovšem s fází posunutou $\pi$.

Hodnotu filtru v daném bodě spočítáme jako součin dvou funkcí, a to sinu (někdy
uváděného ve formě komplexní exponenciály, pokud potřebujeme i komplexní, i
reálnou složku), který slouží jako vlastní signál, a obálky, funkce, která
utlumí signál mimo daný bod. Jako obálka se typicky používá dvojrozměrná
Gaussova křivka. Funkce tedy vypadá jako $$g(x,y) = \exp\left(\frac{x'^2 +
y'^2}{2\rho}\right)\cos\left(2\pi\frac{x'}{\lambda}+\phi\right),$$ kde vektor
$(x',y')^T$ je vektor $(x,y)^T$ otočený o úhel, který svírá osa $x$ se směrem,
podél nějž chceme měřit signál a posunutý do bodu, v němž chceme měřit signál,
$\rho$ je směrodatná odchylka použité Gaussovy křivky (určuje tedy, jak široké
okolí daného bodu nás zajímá), $\lambda$ je frekvence signálu, který hledáme, a
$\phi$ je fázový posun.
   
Gabor filter ale můžeme používat i k samotné tvorbě signálu. Chceme-li vytvořit
v nějakém bodě signál, můžeme spočítat Gabor filter, jako bychom ho tam chtěli
detekovat, a potom ho sečíst se šumem. Takto vytvořenému signálu budeme říkat
Gabor patch.

V této práci se však od nejčastěji používaného Gabor patche
odchýlíme ve dvou bodech. Nebudeme jako obálku používat dvojrozměrnou Gaussovu
funkci, ale tzv. raised cosine. Ve vzorci výše se nám tedy z prvního činitele
stane $\cos(\pi\sqrt{x'^2+y'^2}/r)/2+1$ pro $\sqrt{x'^2+y'^2}\leq r$, jinak 0,
kde $r$ je poloměr signálu. Tuto změnu činíme z implementačních důvodů.
Gaussova křivka totiž nikde nedosáhne nuly, kdežto raised cosine ano. S použitím raised cosine proto stačí při přidávání signálu modifikovat malou část šumu. Vliv na vzhled gabor patche má však tato změna minimální.

Druhý bod, v němž se odchýlíme, bude, že nebudeme činitele násobit a signál
přičítat k šumu, ale budeme se chovat, jako kdybychom kreslili signál na šum a
obálku použijeme jako alfa kanál. Tuto úpravu provedeme proto, že náš šum
musíme nakonec upravit tak, aby všechny jeho hodnoty byly mezi 0 a 255.
Takovýmto ořezáváním bychom mohli, kdybychom přičítali, přijít až o půlku
signálu. 

\section{Ideální bayesovský pozorovatel}
