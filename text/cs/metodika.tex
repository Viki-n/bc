\chapter{Metody}

V této kapitole popíšeme metody, kterými byl prováděn experiment. Technické detaily,
jako například algoritmus, kterým byl generován šum, nebo hodnoceny lokace, jsou
ale uvedeny až v dokumentaci aplikace, která je druhou přílohou této práce.

\section{Účastníci}

Experiment byl proveden celkem s 10 účastníky, pěti v každé skupině. Zrak všech z nich byl
normální nebo upravený na normální (např.  brýlemi). Byli vybráni z okolí
autora této práce, nejednalo se tedy o náhodný výběr z populace. Do skupin byli
rozřazeni náhodně. Ve skupině se zpětnou vazbou byl věkový průměr $26.8$ let,
směrodatná odchylka věku byla $10.1$, dva z účastníků byli muži a tři ženy. Ve
skupině bez zpětné vazby byl průměrný věk 26 let se směrodatnou odchylkou
$11.1$ let, jeden muž a čtyři ženy. Detailnější statistiky jsou uvedeny v první
příloze. Za účast na experimentu nebyli účastníci nijak odměněni a účastnili se
ho dobrovolně. Na začátku experimentu žádný účastník nevěděl nic o vlastnostech
modelů IBO a ELM.

\section{Nástroje a stimuly}
\begin{figure}
\centering
\begin{tabular}{c}
\begin{subfigure}{0.95\textwidth}
\centering
\includegraphics[width = .75\linewidth]{img/noise_visible}
\caption{Šum s dobře viditelným Gabor patchem u levého dolního okraje (kontrast $0.99$).}
\end{subfigure}\\
\noalign{\vskip\bigskipamount}
\\
\begin{subfigure}{0.95\textwidth}
\centering
\includegraphics[width = .75\linewidth]{img/noise_invisible}
\caption{Šum se špatně viditelným Gabor patchem u pravého okraje (kontrast $0.61$).}
\end{subfigure}
\end{tabular}
\caption{Příklady šumů s cíli.}
\label{Sumy}
\end{figure}

K experimentu byla použita aplikace, která je součástí této práce. 
Jako zobrazovací zařízení byl použit iPad air s displejem o rozlišení
$2048\times1536$ a o úhlopříčce $9.7$ palců, což odpovídá rozměrům displeje asi
$19.7 \times 14.8$ centimetrů. Hustota pixelů je 264 pixelů na palec. 

V experimentu použitý růžový šum byl kruhový a jeho průměr byl 1024 pixelů
\footnote{Odsud již všude, kde není zřejmý opak, se pixelem myslí
pixel obrázku, nikoliv pixel displeje.}. Tento průměr byl zvolen proto, že je
to nejbližší mocnina dvojky ke kratšímu rozměr displeje iPadu v pixelech.
Mocnina dvojky byla zvolena kvůli použití rychlé Fourierova transformace při
generování růžového šumu. RMS kontrast\footnote{{\it RMS kontrast} je pro
černobílý obraz vlastně jiný název pro standardní odchylku jasu pixelu, kdy jas
měříme tak, aby černý pixel dostal hodnotu nula a bílý hodnotu jedna
\citep{RMS}.} šumu byl po vygenerování roven $0.25$, ale při zobrazování byly
hodnoty lišící se od střední hodnoty o více než dvě standardní odchylky
přiblíženy ke střední hodnotě právě na tuto vzdálenost, čímž byl RMS kontrast
mírně snížen (pro příklad šumu takového, jaký byl použit viz obrázek \ref{Sumy}). \index{RMS kontrast}

Jako stimulus byl použit Gabor patch. Jedním z problémů aditivního Gabor patche
ale je fakt, že jas pixelu je v praxi omezený. Kdybychom tedy přičítali Gabor
patch k šumu v místě, které má samo o sobě vysoký jas, museli bychom jeho nejvyšší
bod snížit tak, aby součet se šumem nepřesáhl maximální hodnotu jasu pixelu.
Obdobný problém bychom měli s oblastí s nízkým jasem. Tento problém byl vyřešen
tak, že Gabor k šumu nebyl ale přičten k šumu, ale vložen do něj, jako
kdybychom kreslili Gabor patch přes šum a obálka zastupovala alfa kanál. To
znamená, že jas pixelu v bodě $x$ byl spočítán jako $S[x] * (1-o[x]) +
G[x]*o[x]$, kde $S[x]$ je hodnota šumu v bodě $x$, a $G[x]$ a $o[x]$ jsou
hodnoty Gaboru a jeho obálky.  Zvolené parametry Gabor patche byly:

\begin{itemize}
\item Obálka: Raised cosine
\item Průměr: 50 pixelů
\item Frekvence: $1/16$ cyklu na pixel
\item Fázový posun: 0
\item Úhel $\Theta$: $135^\circ$ 
\end{itemize}

\begin{figure}
\centering
\includegraphics[width=0.48\textwidth]{img/locations_outline.png}
\caption {Možné lokace Gabor patche}.
\label{LokaceGP}
\end{figure}

Možných lokací Gabor patche bylo celkem 85, a byly rozmístěny po scéně v
trojúhelníkové mřížce tak, aby jedna možná lokace byla ve středu. Vzdálenost
dvou sousedních možných lokací byla 100 pixelů (viz obrázek \ref{LokaceGP}). 

Kontrast cíle byl daný maximem obálky, tedy při snižování kontrastu byl cíl čím
dál tím průhlednější. Hodnoty kontrastu se mohly pohybovat mezi nulou a
jedničkou.

\section{Procedura}

Každý účastník prošel sadou 3 testů. V prvním testu mu bylo postupně
prezentováno 40 úkolů, kde v každém z nich měl najít Gabor patch v růžovém
šumu.  V druhém testu bylo prezentováno 120 obdobných úkolů a ve třetím opět
40. 

Ve druhém testu dostávaly účastníci, kteří byly ve skupině se zpětnou
vazbou, po každé fixaci zvukovou odpověď, která značila, kolik informace mohli
od této fixace očekávat (tedy jestli bylo z pohledu ELM moudré udělat právě
tuto fixaci). Tato odpověď byla ve formě tónu, jehož frekvence $f$ byla dána
vzorcem $$f = 440\operatorname{Hz}*2^{2-2*\frac{c - \delta}{\Delta -
\delta}},$$ kde $c$ je očekávané snížení entropie při fixaci, kterou si subjekt
vybral, $\Delta$ je maximální dosažitelné očekávané snížení entropie a $\delta$
minimální v případě, že by byla zafixována některá z možných lokací
cíle.\footnote{To znamená, že je potenciálně možné dosáhnout výsledku lepšího
než $\Delta$, resp. horšího, než $\delta$. Rozdíl by však neměl být důležitý.} 

Tento vzorec byl sestaven tak, aby účastník dostal tón a1 v případě, že zvolil
lokaci, která vedla ke stejnému očekávanému snížení entropie, jako lokace,
kterou vybral model ELM jako nejlepší, tón a3 v případě, že naopak zvolil
nějakou z nejhorších možných lokací.

V každém úkolu byl šum překryt černou barvou. Subjekt se měl vždy dotknout
displeje v místě, které se rozhodl zafixovat. Na tomto místě byl poté šum
odkryt na $300 \operatorname{ms}$. Výpočet tvaru a míry odkrytí oblasti bylo
provedeno vynásobením s $d'$ mapou posunutou do bodu fixace, s parametrem
$d'_0$ nastaveným na 1 a ostatními parametry naměřenými na pozorovateli FD
($e_R=223$, $e_L=223$, $e_U = 161$, $e_D = 164$, $\beta=2.46$, všechny
veličiny, u nichž má smysl uvádět jednotku, jsou v pixelech).

Tuto $d'$ mapu\index{d' mapa@$d'$ mapa} by bylo lepší měřit každému účastníkovi
zvlášť. Takové měření ale zahrnuje podle metodiky z článku \citep{Ellipse}
přinejmenším 2000 trialů a s metodikou předepsanými přestávkami trvá celý den.
Aby bylo možné při něm kontrolovat chování účastníka, je navíc potřeba
eyetracker. Proto jsme se rozhodli i vzhledem k nemožnosti kontrolovat přesně
podmínky k tomuto měření nepřistoupit. K tomuto problému se ale znovu vrátíme
závěrem této kapitoly v sekci Limitace.  

Ve chvíli, kdy si účastník myslel, že objevil cíl, zmáčkl tlačítko. Poté mu byl
ukázán celý odkrytý šum, ovšem bez cíle. Potom se měl účastník dotknout šumu na
místě, kde si myslel, že se cíl nacházel. Cíl byl považován za nalezený, pokud
byla vzdálenost vybraného místa a středu skutečné lokace cíle menší než 60
pixelů displeje. Úkol byl považován za úspěšně splněný, pokud byl cíl nalezen a
současně v rámci něj účastník provedl nejvýše šest fixací. 

V každém testu byl počáteční kontrast cíle $0.7$. Pokaždé, když byl účastník
třikrát po sobě úspěšný, byla zvýšena obtížnost snížením kontrastu cíle o
$0.01$, pokud byl třikrát po sobě neúspěšný, byla obtížnost opět snížena. 

Byl tedy využit obecný postup, kterému se říká Up/Down metoda. Tento postup se
používá, pokud je závislost pravděpodobnosti daného jevu na nějakém parametru
rostoucí (či obecně monotónní) funkce. Spočívá v tom, že se daný parametr
snižuje, když jev nastává, a zvyšuje když nenastává (v obou případech o předem
danou konstantu, která se během experimentu nemění, nemusí však být v obou
směrech stejná). Blíže je popsána v knize Psychophysics: A practical
introduction \citep{psychophysics}.  Na rozdíl od implementace této metody,
která je popsaná v knize, jsme se rozhodli za jev považovat tři úspěchy za
sebou a za jeho absenci tři neúspěchy, abychom zmenšili velký rozptyl, který
náš jev jinak má.

Hodnota $0.01$ byla vybrána tak, aby se během 120 úkolů, které jsou v
prostředním testu, účastník mohl teoreticky dostat až na hodnoty kontrastu
kolem $0.3$. S kontrastem menším než přibližně $0.45$ je ale pravděpodobnost
splnění úkolu bez ohledu na strategii podle subjektivního názoru autora práce
velmi nízká. V experimentu se tento dojem potvrdil, nejnižší v experimentu
dosažený kontrast byl $0.5$.



\section{Limitace}

Při návrhu experimentu jsme narazili na několik problémů, které mohou vnést
nepřesnosti do měření, ale jejichž řešení je mimo rozsah této práce. Konkrétně
se jedná o následující obtíže:

\begin{itemize}
\item Vjemy lidského pozorovatele neodpovídají příliš dobře vjemům simulovaného
ideálního pozorovatele. Aby si tyto vjemy odpovídaly, alespoň přibližné, museli
bychom každému pozorovateli změřit jeho vlastní $d'$ mapu. Měření $d'$ mapy ale
i  v té nejminimalističtější variantě, která se používá, trvá nejméně jeden
pracovní den. Druhým důvodem, proč mohou být vjemy rozdílné i v případě, že by
konstanty $d'$ mapy vyšly pozorovateli stejně, jaké byly použity, je, že tato
naměřená $d'$ mapa odpovídá situaci, kdy je scéna se šumem umístěna tak daleko
od pozorovatele, aby ji viděl pod zorným úhlem $15^\circ$. To při velikosti
scény v našem případě odpovídá vzdálenosti pozorovatele a zařízení přibližně
$65 \operatorname{cm}$. V našich experimentech nebyla vzdálenost pozorovatele od
scény hlídána a určitě byla nižší, než řečených $65 \operatorname{cm}$
(dodržení této vzdálenosti by odpovídalo situaci, kdy by účastníci držely iPad
před sebou zhruba na délku natažené paže). 

\item I pokud odhlédneme od nepřesností zmíněných v předchozím bodě a dovolíme
si na chvíli (evidentně scestný) předpoklad, že účastnící měly vlastní $d'$ mapu
konstantní, narazíme na další problém. Okraj oblasti, která byla odkrývána,
byl ztmavován lineárně se snižující se hodnotou $d'$ v použité $d'$ mapě.
Závislost $d'$ na kontrastu ale téměř jistě
není lineární.

\item S tím souvisí ještě jeden problém: Subjekty samozřejmě nemají svou $d'$
mapu konstantní. Tato mapa se tedy nějak skládá s $d'$ mapou, pomocí které
bylo určeno odhalování šumu. V práci jsme toto skládání ignorovali (tedy
předpokládali jsme, že $d'$ na kontrastu závisí lineárně a $d'$ mapa účastníků
je konstantní.) Nabízela by se otázka, proč tedy bylo zatemňování šumu vůbec
prováděno. To se dělo z několika důvodů:

\begin{itemize}

\item Zatemňování šumu zavádí potřebu klikat na místa, která chce pozorovat v
dalším kroku prozkoumat. Nutí tedy pozorovatele, aby tato rozhodnutí dělal
vědomě a nikoli podvědomě, což byl jeden z efektů, které jsme chtěli zkoumat.

\item Celý proces jedné fixace tímto způsobem také trvá mnohem déle (nižší
jednotky vteřin místo nižších desetin vteřiny) a poskytuje nám tedy mnoho času
na update mapy aposteriorních pravděpodobností a výpočet množství informace,
kterou lze získat následující fixací.

\item Takto navržený experiment též umožňuje zjišťovat, které lokace účastník
fixuje bez použití eyetrackeru nebo jiných technologií.

\end{itemize}

\item Je možné, že u lidí, kteří nikdy dříve Gabor patch hledat nezkoušeli, se
během testů mění jejich senzitivita na Gabor patch (mění se hodnota $d'(0,0)$).

\item Vzhledem k tomu, že počet reálných pixelů displeje neodpovídal (a ani
nebyl dělitelný) velikostí scény v pixelech, je možné, že byly efekty jako
například antialiasing změněny lokální kontrasty scény.

\item Ve výzkumu v oblasti psychofyziky se většinou
pečlivě kontroluje prostředí (například se zatemňuje místnost, v níž se provádí experiment). To jsme v našem výzkumu nedělali.

\end{itemize}

Tyto nepřesnosti však nepovažujeme za příliš důležité, protože cílem této práce
není přesný experiment, ale pouze proof of concept. Za zásadní chybu naopak
nepovažujeme malý počet účastníků -- kdybychom chtěli na této práci postavit
přesný experiment, nebylo by potřeba zvyšovat počet účastníků, ale pouze počet
měření na jednotlivých účastnících, například tak, že bychom vícekrát opakovali
první a třetí test. Jinak ale není ve výzkumu nutně přínosné velký vzorek,
často je lepší na malém vzorku provést větší počet přesnějších měření
\citep{SmallN}. Kdybychom se však rozhodli raději pro Large-$N$ design, mohli
bychom většinu ostatních parametrů experimentu ponechat, ale bylo by potřeba
přinejmenším o řád více účastníků. 

