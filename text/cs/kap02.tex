\chapter{Metody}

\TODO{Nemělo by být napřed formálně posáno, co to vlastně zkoumáme, než začneme popisovat metody?}

\section{Účastníci}

\section{Nástroje a stimuly}

K experimentu byla použita aplikace, která je součástí této práce. 
Jako zobrazovací zařícení byl použit iPad air s displejem o rozlišení
$2048\times1536$ a o úhlopříčce $9.7$ palců, což odpovídá rozměrům displeje asi
$19.7 \times 14.8$ centimetrů. Hustota pixelů je 264 pixelů na palec. 

V experimentu použitý růžový šum byl kruhový a jeho průměr byl 1024 pixelů.
\footnote{Odsud již všude, kde není zřejmý opak, se pixelem myslí
pixel obrázku, nikoliv pixel displeje.} Tento průměr byl zvolen proto, že je
to nejbližší mocnina dvojky ke kratšímu rozměr displeje iPadu v pixelech.
Mocnina dvojky byla zvolena kvůli použití rychlé Fourierova transformace při
generování růžového šumu.

Jako stimulus byl použit Gabor patch. Nebyl ale přičten k
šumu, ale vložen do něj, jako kdybychom kreslili Gabor patch přes šum a obálka
zastupovala alfa kanál.  Zvolené parametry Gabor patche byly:

\begin{itemize}
\item Obálka: Raised cosine
\item Průměr: 50 pixelů
\item Frekvence: $1/16$ cyklu na pixel
\item Fázový posun: 0
\item Úhel $\Theta$: $135^\circ$ 
\end{itemize}

Možných lokací Gabor patche bylo celkem 85, a byly rozmístěny po scéně v
trojúhelníkové mřížce tak, aby jedna možná lokace byla ve středu. Vzdálenost
dvou sousedních možných lokací byla 100 pixelů. 

Kontrast cíle byl daný maximem obálky, tedy při snižování kontrastu byl cíl čím dál tím průhlednější. Hodnoty kontrastu se mohly pohybovat mezi nulou a jedničkou.

\section{Procedura}

Každý subjekt podstoupil napřed měření kontrastu, který musí gabor patch mít, aby
pravděpodobnost, že bud detekován na středu zorného pole byla přibližně 90
\%\TODO{Budeme to měřit na všech?}. Následně každý subjekt prošel sadou 3
testů. V prvním testu mu bylo postupně prezentováno 50 úkolů, kde
v každém z nich měl najít Gabor patch v růžovém šumu. Úkol byl považován za
úspěšný, pokud byl cíl nalezen během pěti či méně fixací. V druhém testu bylo prezentováno 150 obdobných úkolů a ve třetím opět 50. Ve druhém testu dostávaly
subjekty, které byly ve skupině se zpětnou vazbou, po každé fixaci zvukovou
odpověď, která značila, kolik informace mohli od této fixace očekávat (tedy
jestli bylo z pohledu IBO/ELM moudré udělat právě tuto fixaci). Tato odpověď
byla ve formě tónu, jehož frekvence byla dána vzorcem
$$\text{\TODO{Vymyslet}},$$ kde $\Delta$ je maximální dosažitelné očekávané
snížení entropie a $\delta$ minimální v případě, že by byla zafixována některá
z možných lokací cíle.\footnote{To znamená, že je potenciálně možná dosáhnout
výsledku lepšího než $\Delta$, resp. horšího, než $\delta$. Rozdíl by však
neměl být důležitý. Meze jsou do vzorce přidány, abychom zabránili absurdním
hodnotám v případě, kdy by si hodoty $\Delta$ a $\delta$ byly téměř rovny, ale
nějaký bod, který není možnou lokací, by měl očekávané snížení entropie o
trochu vyšší či nižší.} Pokaždé, když byl subjekt třikrát po sobě úspěšný,
byla zvýšena obtížnost snížením kontrastu cíle \TODO{O kolik?}, pokud byl
třikrát po sobě neúspěšný, byla obtížnost opět snížena.

V každém úkolu byl šum překryt černou barvou. Subjekt se měl vždy dotknout displeje v místě, které se
rozhodl zafixovat. Na tomto místě byl poté šum odkryt na $300
\operatorname{ms}$. Výpočet tvaru a míry odkrytí oblasti bylo provedeno
vynásobením s $d'$ mapou posunutou do bodu fixace, s parametrem $d'_0$
nasteveným na 1 a ostatními parametry naměřenými na pozorovateli FD
($e_R=223$, $e_L=223$, $e_U = 161$, $e_D = 164$, $\beta=2.46$, všechny
veličiny, u nichž má smysl uvádět jednotku, jsou v pixelech).

Ve chvíli, kdy si subjekt myslel, že objevil cíl, zmáčkl tlačítko. Poté mu byl ukázán celý odkrytý šum, ovšem
bez cíle. Potom se měl subjekt dotknout šumu na místě, kde si myslel, že se cíl
nacházel. Cíl byl považován za nalezený, pokud byla vzdálenost vybraného místa
a středu skutečné lokace cíle menší než 50 pixelů. 

\section{Limitace}

Při návrhu experimentu jsme narazili na několik problémů, které mohou vnést
nepřesnosti do měření, ale jejichž řešení je mimo rozsah této práce. Konkrétně
se jedná o následující obtíže:

\begin{itemize}
\item Vjemy lidského pozorovatele neodpovídají příliš dobře vjemům simulovaného
idálního pozorovatele. Aby si tyto vjemy odpovídaly, alespoň přibižně, museli
bychom každému pozorovateli změřit jeho vlastní $d'$ mapu. Měření $d'$ mapy ale
i  v té nejminimalističtější variantě, která se používá, trvá nejméně jeden
pracovní den. Druhým důvodem, proč mohou být vjemy rozdílné i v případě, že by
konstatnty $d'$ mapy vyšly pozorovateli stejně, jaké byly použity, je, že tato
naměřená $d'$ mapa odpovídá situaci, kdy je scéna se šumem umístěna tak daleko
od pozorovatele, aby ji viděl pod zorným úhlem $15^\circ$. To při velikosti
scény v našem případě odpovídá vzdálenosti pozorovatele a zařízení přibližně
$65 \operatorname{cm}$. V našich experimentech nebyla vzdálenost pozorovatele od
scény hlídána. a určitě byla nižší, než řečených $65 \operatorname{cm}$
(dodržení této vzdálenosti by odpovídalo situaci, kdy by subjěkty držely iPad
před sebou zhruba na délku natažené paže). 

\item I pokud odhlédneme od nepřesností zmíněných v předchozím bodě a dovolíme
si na chvíli (evidentně scestný) předpoklad, že subjekty měly vlastní $d'$ mapu
konstatntní, narazíme na další problém. Okraj oblasti, která byla odkrývána,
byl ztmavován lineárně se snižující se hodnotou $d'$ v použité $d'$ mapě.
Závislost $d'$ na kontrastu ale téměř jistě
není lineární.

\item S tím souvisí ještě jeden problém: Subjekty samozřejmě nemají svou $d'$
mapu konstatntní. Tato mapa se tedy nějak skládá s $d'$ mapou, pomocí které
bylo určeno odhalování šumu. V práci jsme toto skládání ignorovali (tedy
předpokládali jsme, že $d'$ na kontrastu závisí lineárně a $d'$ mapa subjektů
je konstatntní.) Nabízela by se otázka, proč tedy bylo zatemňování šumu vůbec
prováděno. To se dělo z několika důvodů:

\begin{itemize}

\item Zatemňování šumu zavádí potřebu klikat na místa, která chce pozorovat v
dalším kroku prozkoumat. Nutí tedy pozorovatele, aby tato rozhodnutí dělal
vědomě a nikoli podvědomě, což byl jeden z efektů, které jsme chtěli zkoumat.

\item Celý proces jedné fixace tímto způsobem také trvá mnohem déle (nižší
jednotky vteřin místo nižších desetin vteřiny) a poskytuje nám tedy mnoho času
na update mapy posteriorních pravděpodobností a výpočet množství informace,
kterou lze získat následující fixací.

\item Takto navržený experiment též umožňuje zjišťovat, které lokace subjekt
fixuje bez použití eyetrackeru nebo jiných technologií.

\end{itemize}

\item Je možné, že u lidí, kteří nikdy dříve Gabor patch hledat nezkoušeli, se
během testů mění jejich sensitivita na Gabor patch (mění se hodnota $d'(0,0)$).

\item Vzhledem k tomu, že počet reálných pixelů displeje neodpovídal (a ani
nebyl dělitelný) velikostí scény v pixelech, je možné, že byly efekty jako
například antialiasing změněny lokální kontrasty scény.

\item Ve výzkumu týkajícím se zkoumání vizuálního vyhledávání se většinou
pečlivě kontroluje prostředí (například se zatemňuje místnost, v níž se provádí experiment). To jsme v našem výzkumu nedělali.

\end{itemize}

Tyto nepřesnosti však nepovažujeme za příliš důležité, protože cílem této práce
není přesný experiment, ale pouze proof of concept. Za zásadní chybu naopak
nepovažujeme malý počet účastníků -- kdybychom chtěli na této práci postavit
přesný experiment, nebylo by potřeba zvyšovat počet účastníků, ale pouze počet
měření na jednotlivých účastnících, například tak, že bychom vícekrát opakovali
první a třetí test. Jinak ale není ve výzkumu nutně přínosné velký vzorek,
často je lepší na malém vzorku provést větší počet přesnějších měření
\citep{SmallN}.

