\chapter{Metody}



\section{Limitace}

Při návrhu experimentu jsme narazili na několik problémů, které mohou vnést
nepřesnosti do měření, ale jejichž řešení je mimo rozsah této práce. Konkrétně
se jedná o následující obtíže:

\begin{itemize}
\item Vjemy lidského pozorovatele neodpovídají příliš dobře vjemům simulovaného
idálního pozorovatele. Aby si tyto vjemy odpovídaly, alespoň přibižně, museli
bychom každému pozorovateli změřit jeho vlastní $d'$ mapu. Měření $d'$ mapy ale
i  v té nejminimalističtější variantě, která se používá, trvá nejméně jeden
pracovní den. Druhým důvodem, proč mohou být vjemy rozdílné i v případě, že by
konstatnty $d'$ mapy vyšly pozorovateli stejně, jaké byly použity, je, že tato
naměřená $d'$ mapa odpovídá situaci, kdy je scéna se šumem umístěna tak daleko
od pozorovatele, aby ji viděl pod zorným úhlem $15^\circ$. To při velikosti
scény v našem případě odpovídá vzdálenosti pozorovatele a zařízení přibližně
$65 \operatorname{cm}$. V našich experimentech nebyla vzdálenost pozorovatele od
scény hlídána.

\item Okraj
\end{itemize}
