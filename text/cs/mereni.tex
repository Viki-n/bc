\chapter{Výsledky}

V~této kapitole stručně představíme výsledky měření. Všechny výsledky
prezentované v~této kapitole jsou založeny pouze na měření z~prvního a třetího
testu. Data z~prvních pěti úkolů prvního testu navíc nepoužijeme, protože se
během nich účastníci ještě seznamovali s~rozhraním aplikace, s~jejíž pomocí byl
experiment prováděn. Ve všech případech byly použity výsledky všech testů, ne
jen těch, v~nichž byl cíl nakonec úspěšně nalezen. Grafy, které jsou vytvořeny
pouze z~dat z~úkolů, ve kterých byl cíl nalezen, však vypadají velmi podobně.

Svislé čáry v~grafech značí intervaly spolehlivosti na konfidenční hladině $95\,\%$.

\def\graphfigure#1#2#3{
\begin{figure}[h!]
\centering
 \makebox[\textwidth][c]{\includegraphics{graphs/#1}}
\caption{#2}
\label{#3}
\end{figure}
}

\graphfigure{predpo_grid}{Graf znázorňující průměrný počet fixací pro jednotlivé účastníky v~jednom úkolu před a po učení.}{beforeall}

\long\def\komentar#1{}

První graf (viz obrázek \ref{beforeall}) zobrazuje výsledky účastníků v~prvním a ve třetím testu. Na grafu je vidět, že téměř všichni účastníci se tréninkem zlepšili, ale není z~něj dobře vidět průměrné zlepšení a vůbec není vidět porovnání mezi skupinami.

\komentar{
Protože grafy průměrného počtu fixací před a po učení (viz obrázky \ref{beforeall} a \ref{afterall}) samy o~sobě nevypadají příliš porovnatelně, v~tabulce jsou uvedené jednotlivé průměry a jejich rozdíly.

\begin{center}
\begin{tabular}{cccc}
\hline
\hline
\multirow{2}{*}{Účastník} & \multicolumn{2}{c}{Průměrný počet fixací}&\multirow{2}{*}{Zlepšení} \\
&Před učením & Po učení &\\ 
\hline
Kontrolní 1       &  4.06  &3.62 & 0.432\\
Kontrolní 2     &    4.03 & 5.88 &-1.85\\ 
Kontrolní 3      &   4.23 & 4.35 &-0.121\\
Kontrolní 4    &     5.46 & 3.65 & 1.81 \\
Kontrolní 5     &    5.69 & 4.35 & 1.34 \\
Experimentální 1      &   4.74 & 4.08 & 0.668\\
Experimentální 2       &  4.74  &4.97 &-0.232\\
Experimentální 3       &  5.51 &  3.7  & 1.81 \\
Experimentální 4     &   5.63  &3.8  & 1.83 \\
Experimentální 5      &  7.31 & 4.28  &3.04 \\

\hline
\hline
\end{tabular}
\end{center}
}

\graphfigure{total_all}{Graf znázorňující průměrný počet fixací na úkol v~rámci skupin v~prvním a třetím testu}{total}

Tyto hodnoty vizualizuje druhý graf (viz obrázek \ref{total}). Zlepšení
u~experimentální skupiny je větší než u~kontrolní skupiny, ale není jasné, do
jaké míry je tento rozdíl způsobený tím, že experimentální skupina byla na začátku
horší než skupina kontrolní. 

\begin{center}
\begin{tabular}{cccc}
\hline\hline
       Efekt & Testová statistika$^1$&$p$-hodnota & $\eta^2_p$ \\
\hline                                            
       Skupina&   1.14& 0.318 & 0.12 \\ 
       Test &   4.23& 0.074 & 0.35 \\ 
 Skupina:Test&   1.69& 0.230 & 0.17 \\ 
\hline\hline
\multicolumn{4}{l}{\footnotesize \textit{Pozn:}
$^1$ Testová statistika pro mixed ANOVA. Stupně volnosti jsou 1, 8.}

\end{tabular}
\end{center}

Z~hodnot $\eta^2_p$ uvedených v~tabulce vyplývá, že učení jako takové má větší vliv na výkon, než příslušnost k~jednotlivým skupinám, žádné z~těchto výsledků ale nejsou statisticky významné. Když se však podíváme na kontrast, na kterém se účastníci při experimentu pohybovali, dostaneme zajímavější výsledky. V~tomto porovnání již vliv učení dosahuje statistické významnosti, což ilustruje graf (viz obrázek \ref{diff}). V~následující tabulce pak jsou přesné hodnoty.

\begin{center}
\begin{tabular}{cccc}
\hline\hline
       Efekt & Testová statistika$^1$&$p$-hodnota & $\eta^2_p$ \\
\hline                                         
       Skupina&       1.03& 0.341&   0.11    \\
      Test&     19.19& 0.002&  0.71  \\
 Skupina:Test&      1.75& 0.223&  0.18       \\

\hline\hline
\multicolumn{4}{l}{\footnotesize \textit{Pozn:}
$^1$ Testová statistika pro mixed ANOVA. Stupně volnosti jsou 1, 8.}


\end{tabular}
\end{center}


\graphfigure{difficulty_average}{Graf znázorňující obtížnost, na které se účastníci v~jednotlivých skupinách a fázích v~průměru pohybovali}{diff}


Detailní grafy ke každému pozorovateli zvlášť jsou uvedeny v~první příloze této práce.
Data a program jazyka R použitý k~tvorbě těchto (a mnoha dalších v~práci nakonec neuvedených) grafů je součástí
elektronických příloh této práce. Též je dostupný na Open Science Frameworku na stránce {\tt https://osf.io/h7ctf/}.

