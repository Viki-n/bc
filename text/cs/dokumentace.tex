\openright
\sekcicka{Druhá příloha -- Dokumentace aplikace}

V~této příloze stručně pojednáme o~aplikaci, která je součástí této práce. 

\subsection*{Uživatelská dokumentace}

Při spuštění aplikace se uživatel dostane na úvodní stránku, kde lze nastavit
jméno účastníka. Z~ní lze příslušnými tlačítky pokračovat k~vlastnímu
testování, do nastavení nebo na stránku s~popisem úkolu, který účastníka
experimentu čeká.

Na stránce s~popisem úkolu si uživatel může přečíst, jak bude probíhat experiment. V~horní části
obrazovky je vždy stručný popis části úkolu a pod ní je šum, který tuto část
úkolu ilustruje. K~další části přejde kliknutím kamkoli na obrazovku. Délka
ukázky se liší podle toho, zda má aktuální uživatel zapnutou zpětnou vazbu --
v~takovém případě jsou na konec přidány ještě čtyři části popisující zvuk. Po
poslední instrukci je uživatel vrácen na obrazovku, ze které se na stránku
s~popisem úkolu dostal (tj. buď na úvodní stránku, nebo do nastavení).

Na stránce s~nastavením je možné měnit mnoho různých parametrů vyhledávání. Pro
replikaci experimentu tak, jak byl prováděn v~této práci, stačí pouze nastavit
jméno, zpětnou vazbu a v~ostatních parametrech nechat výchozí hodnoty.
Následující tabulka však nabízí popis toho, co přesně které nastavítko ovládá.
U~všech nastavítek, u~nichž se ovládá velikost něčeho,
se hodnota nastavuje v~pixelech obrázku.
\bigskip

{
\newcolumntype{R}[1]{>{\raggedright\let\newline\\\arraybackslash\hspace{0pt}}m{#1}}
\renewcommand{\arraystretch}{2}
\rowcolors{2}{gray!25}{white}
\begin{longtable}{R{0.25\textwidth}p{0.7\textwidth}}
\hline
\hline
\rowcolor{white}
Popisek & Význam \\
\hline
\endfirsthead
\hline
\hline
\rowcolor{white}
Popisek & Význam \\
\hline
\endhead
\hline
\endfoot
\hline
\hline
\endlastfoot

Difficulty & Průhlednost Gabor patche vynásobená tisícem a zaokrouhlená na
jednotky. Editací tohoto pole se změní jak současná obtížnost, tak obtížnost,
na kterou se hledání vrátí na začátku druhého a třetího testu. Hodnota v~poli
při vstupu do nastavení odpovídá první zmíněné obtížnosti.\\

Gabor patch diameter & Průměr obálky Gabor patche. Perioda se automaticky
nastaví na třetinu této hodnoty. \index{Gabor patch}\\

Radius of uncovered area & V~případě, že je vypnuté použití $d'$ mapy\index{d'
mapa@$d'$ mapa}, odkrývá se šum pomocí masky Raised cosine. Zde je možné
nastavit poloměr masky v~tomto případě.\\ 

Area uncovered for (ms) & Čas v~milisekundách, na který se odkryje šum při
doteku.\\

Area uncovered for brief period only & Určuje, zda se šum odkrývá na trvalo (a
jednotlivé doteky se zobrazují současně), nebo pouze na dobu nastavenou
v~předchozím bodě. Při odkrývání natrvalo se však bez ohledu na nastavení vždy
použije k~odkrývání maska Raised cosine.\\

Sound fully relative & Mění způsob, kterým se počítá výška zvuku. Při vypnutém
nastavítku se používá vzorec uvedený v~kapitole Metody. Při zapnutém se
spočítá, kolik možných lokací cíle má z~pohledu ELM vyšší hodnotu, než vybrané
místo. Tón se pak vybere tak, aby jeho frekvence byla mezi $440$ a $1760$
Hertzy a byla exponenciálně závislá na tomto počtu. \\

Possible location distance & Vzdálenost možných lokací cíle \\

Regenerate noise & Návrat k~měření se zapnutou touto možností způsobí kompletní
restart současného úkolu, včetně přegenerování šumu a lokace Gabor patche.
Vhodné zapnout, pokud jsme měnili některá jiná nastavení. \\

Preview & Tlačítko, které přesune uživatele na stránku s~popisem úkolu. \\ 

Start game & Tlačítko, které zobrazí obrazovku s~experimentem. \\

Back to main menu & Tlačítko sloužící k návratu na úvodní obrazovku. \\

Trials in first and third test & Počet úkolů  v~prvním a třetím testu. \\

Trials in second test & Počet úkolů během tréninku. \\

Correct/wrong responses in row before difficulty changes & Počet správných či
špatných odpovědí v~řadě, po kterých se změní obtížnost. \\

Difficulty changed by & O~kolik se změní obtížnost (tisícem vynásobená
průhlednost Gabor patche), když se má měnit (viz předchozí řádek). \\

Accuracy threshold & Maximální chyba v~pixelech displeje, o~kterou se uživatel
může splést, aby byl úkol ještě uznán jako úspěšně splněný. \\

Maximal amount of fixations for succesfull trial & Maximální počet fixací, po
nichž musí být oznámeno nalezení cíle, aby byl úkol považován za úspěšně
splněný. \\

Calculate score? & Vynuluje a zapne počítání a zobrazování skóre. Skóre získané
jedním úkolem je 0, pokud cíl nebyl nalezen, jinak je určeno podle vzorce
$\lfloor(1000 - \text{(Obtížnost)})/\text{(Počet fixací)}\rfloor$. \\

$d'$ map constants settings & Prvních šest nastavení v~této kategorii určuje
přímočarým způsobem konstanty $d'$ mapy tak, jak jsou popsány v~kapitole, která
se jí věnuje. Hodnoty $d'_0$ a $\beta$ je nutné zadávat vynásobené stem. \\

Recalculate $d'$ map & Po změně parametrů $d'$ mapy přepočítá pomocné datové
struktury. Přepočítávání se neděje automaticky po změně libovolné
hodnoty, protože může trvat až několik vteřin a často je potřeba změnit více
konstant současně. Struktury se v případě, že byly konstanty změněny ale toto
tlačítko nebylo stlačeno, přepočítají automaticky při opuštění nastavení.  \\

Use $d'$ map & Rozhoduje o~tom, zda bude pro odkrývání šumu použita maska
spočítaná z~$d'$ mapy, nebo Raised cosine.\\

Name & Identifikátor účastníka.\\

Feedback & Zapíná a vypíná zpětnou vazbu pro účastníka v~druhém testu.
V~případě, že je jméno účastníka \uv{unknown}, což je defaultní hodnota, je při
zapnutí této funkce z~ladících a předváděcích důvodů zpětná vazba poskytována
bez ohledu to, v~kterém testu se nachází.\\

Measure detectability & Spustí obrazovku, kde je uživateli prezentována řada
pokusů, kdy se vždy Gabor patch se současnými parametry objeví ve středu
s~pravděpodobností $1/2$. Uživatel pak může otestovat, jakou má při daných
vlastnostech cíle hit rate a false alarm rate.\\

See log & Zobrazí log všech testů od posledního vymazání všech dat. Log zůstává
napříč spuštěními aplikace a ukládá se automaticky po každém dokončeném
úkolu.\\

Data deletion enabled & Je-li tato možnost aktivována, stlačení tlačítka
\uv{Delete all data} opravdu smaže všechna data. V~opačném případě je uživatel
upozorněn na to, že musí mazání dat nejprve povolit.\\

Delete all data & Viz předchozí řádek.\\

Generate noise & V~případě, že je tato funkce vypnutá, bude místo růžového šumu
vygenerována pouze šedá plocha.\\

Display locations & Je-li tato možnost zapnuta, bude v~místě každé lokace
zobrazen světlý kruh. Přesný odstín jeho pravé horní poloviny je závislý na
aposteriorní pravděpodobnosti, kterou ELM pozorovatel přiřazuje právě této
lokaci (čím světlejší půlkruh, tím vyšší pravděpodobnost). Odstín levé dolní
poloviny je závislý na očekávaném snížení entropie, které podle ELM
pozorovatele nastane, zafixujeme-li v~dalším kroku tuto lokaci (opět světlejší
kruh znamená větší očekávané snížení entropie). Ve středu kruhu, kde je
očekávané snížení entropie největší (ve všech takových, není-li to jednoznačné)
se nachází černá tečka. \\

\end{longtable}
 }

\bigskip
V~obrazovce, kde se provádí experiment, se vše koná tak, aby jediný nutný zásah
do experimentu od administrátora byl, že bude sledovat počet uskutečněných
úkolů (napsaný spolu s~dalšími důležitými hodnotami v~horní části obrazovky) a
upozorní účastníka, když má experiment končit. Aplikace sama postupně mění
obtížnost, dává zvukovou zpětnou vazbu v~situacích, kdy má, a restartuje
obtížnost na počáteční hodnotu na začátku druhého a třetího úkolu.

\subsection*{Programátorská dokumentace}

Aplikace je napsána v~jazyce Swift, který je určený pro psaní nativně běžících
aplikací pro operační systém iOS. Je určená na zařízení iPad air.

V~této sekci stručně přiblížíme dva netriviální algoritmy, které se v~aplikaci
vyskytují. Bude se jednat o~výpočet očekávaného snížení entropie v~rámci
simulace ELM pozorovatele a generování růžového šumu.

V~obou těchto algoritmech je potřeba generovat náhodná čísla s~normálním
rozdělením. Na to použijeme funkci {\tt GKGaussianDistribution.nextUniform()},
která je součástí knihovny {\tt GameplayKit}, jedné ze standardních knihoven
jazyka Swift. Tato funkce se od normálního rozdělení odklání v~tom, že
nepřipouští hodnoty, které se od střední hodnoty rozdělení liší o~více než tři
standardní odchylky. Protože je však pravděpodobnost takových hodnot
v~normálním rozdělení rovna přibližně $0.003$, rozhodli jsme se, že ovlivnění
experimentu touto odlišností zanedbáme.

\subsubsection*{Model ELM pozorovatele}

ELM \index{ELM pozorovatel} byl v~práci vytvořen přesně tak, jak byla jeho simulace popsána v~článku \citep{Najemnik09}. 

To znamená, že při fixaci bodu $x$  dostane z~každé lokace $y$ jako pozorování reálné
číslo, které je náhodně vybrané z~normálního rozdělení, jehož střední hodnota
je $-0.5$, pokud se cíl v~daném bodě nenachází, a $0.5$, pokud ano. Směrodatná
odchylka tohoto rozdělení je rovna hodnotě $d'(y-x)$\index{d'@$d'$}. Poté
spočítá veškerou informaci, kterou již o~jednotlivých lokacích dostal, jako
součet odpovědí, které dostal při předchozích fixacích, vynásobený druhými
mocninami příslušných hodnot $d'$ (ve skutečnosti stačí přičíst odpovědi
z~aktuální fixace k~již spočítanému součtu po předchozí fixaci).

V~dalším kroku spočítá aposteriorní pravděpodobnosti prostě tak že pro získaný
součet $s$ nastaví aposteriorní pravděpodobnost na $e^s$, a pak ještě všechny
aposteriorní pravděpodobnosti vydělí jejich součtem, aby byl po vydělení součet
roven jedné. Až po sem je algoritmus lineární v~počtu možných lokací. Nyní je
však potřeba spočítat očekávané snížení entropie. To se spočítá podle již dříve
uvedeného vzorce jako $$\displaystyle\sum_{i\in L} p_T(i)d'^2(i-k(T+1)),$$ kde
$L$ je množina možných lokací, $p_T(i)$ je aposteriorní pravděpodobnost, že se
cíl nachází v~lokaci $i$ a $k(T+1)$ je lokace, jejíž očekávané snížení entropie
počítáme.   

Před první fixací se očekávaná snížení entropie počítají obdobně, až na to, že
jako pole s~předchozími odpovědi použijeme pole vyplněné samými nulami.

Důkaz korektnosti algoritmu zde nebudeme uvádět, neboť je již uvedený\linebreak v~článcích \citep{Najemnik05, Najemnik08, Najemnik09}.

\subsubsection*{Generování růžového šumu}

\index{Šum!růžový} 

Šum generujeme tak, že nejprve vygenerujeme dvojrozměrné pole náhodných
komplexních čísel stejně velké, jak velký chceme výsledný šum. Tato komplexní
čísla mají v~goniometrické podobě rovnoměrně náhodné argumenty. Absolutní
hodnotu čísla na pozici $(x,y)$ v~poli $n\times n$ vybereme z~normálního
rozdělení se střední hodnotou \begin{equation}\label{Pink}\left(\min\left(x,n-x\right)^2
+\min\left(x,n-x\right)^2\right)^{-2}\end{equation} a směrodatnou odchylkou rovnou třetině
této hodnoty. V~případě, kdy by tato hodnota vyšla jako nekonečno, můžeme
hodnotu prostě zahodit (zdůvodnění tohoto tvrzení dále vyplyne z~algoritmu) a
použít libovolnou. Výběr střední hodnoty vychází z~definice růžového šumu,
odchylka byla zvolena při testování kvůli výše uvedeným vlastnostem náhodného
generátoru (nemůže se nám stát, že by vyšla záporná absolutní hodnota) a byla
na této hodnotě ponechána, protože algoritmus dával vizuálně žádoucí výsledky a
definice růžového šumu ji nespecifikuje.

Na takto vytvořené pole zavoláme (zpětnou, ale ve skutečnosti na tom nezáleží)
dvojrozměrnou rychlou Fourierovu transformaci.\footnote{Dvojrozměrná diskrétní
Fourierova transformace je Fourierova transformace nejprve provedená po řádcích
a potom po sloupcích.}  Výsledkem je růžový šum. My ho ale potřebujeme být
schopni zobrazit. Nejprve tedy zahodíme imaginární složku komplexních čísel.
Poté spočítáme jeho průměr a směrodatnou odchylku a uplatníme na něj lineární
transformaci takovou, aby jeho střední hodnota byla 128 a směrodatná odchylka
64. Nakonec záporné hodnoty nastavíme na nulu a hodnoty převyšující 255 na 255.
Nakonec nastavíme na 128 ty pixely, jejichž (eukleidovská) vzdálenost od středu
obrazu je větší než $n/2$, čímž dosáhneme toho, že výsledné pole bude kruhové.
Výsledné pole interpretujeme jako černobílý obraz s~osmi bity na pixel.


Fourierovu transformaci nalezneme v~knihovně {\tt Accelerate}, která je jednou ze
standardních knihoven jazyka Swift.

Nyní se vrátíme k~tomu, proč jsme mohli zahodit hodnotu, pro níž vyšel
nekonečný koeficient. Ten vyjde právě tehdy, když jsou ve vzorci \eqref{Pink}
obě sčítané závorky nulové. Budeme-li $x$ i $y$ indexovat od nuly, znamená to,
že musí být $x$ i $y$ rovno nule. To znamená, že se tato situace týká jednoho
jediného bodu, a to bodu $(0,0)$. Protože ale poté voláme Fourierovu
transformaci, z~nultého bodu se stane hodnota, která se násobí pouze jedničkou,
tedy ji beze změny přičteme ke každé hodnotě. Protože ale potom stejně šum
lineárně transformujeme, aby měl správnou střední hodnotu a směrodatnou
odchylku, žádná předchozí nekonstantní lineární transformace výstup neovlivní
(nebo alespoň ne více než tím, že způsobí nějaké nechtěné
zaokrouhlování. V algoritmu je komplexní číslo totiž reprezentováno pomocí
dvou hodnot typu {\tt Double}. To ale nebude mít v~praxi významný vliv).

Pro generování jiného z~šumů, o~kterých se mluví v~sekci o~barvách šumu, stačí 
ve vzorci \eqref{Pink} vynásobit hodnotu $-2$ v~exponentu příslušnou hodnotou $\beta$.
