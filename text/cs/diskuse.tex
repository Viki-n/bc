\chapter{Diskuse}

Cílem experimentu bylo ukázat, zda má zpětná vazba vliv na to, jak dobře se
člověk během daného počtu pokusů naučí hledat cíl.

Analýza naznačuje, že během 120 tréninkových pokusů dochází k měřitelnému a
statisticky významnému zlepšení. Pokud budeme jako jeho ukazatel brát průměrnou
obtížnost, zjistíme, že velikost efektu $\eta^2_p$ dosáhla hodnoty $0.71$,
přičemž velikost daného jevu se hodnotí jako velká od hodnoty $\eta^2_p=0.14$
\citep{Cohen}. Zlepšení je tak výrazné, že i přes relativně malý počet vzorků
vyšla $p$-hodnota $0.002$, což znamená, že hypotézu, že by k učení vůbec
nedocházelo, můžeme zavrhnout nejen na $95\%$ konfidenční hladině, ale i na
$99\%$. Zpětná vazba má na učení nejspíš vliv také. Dosavadní data dokonce
naznačují, že stále velký, a to bez ohledu na to, zda jako ukazatel zlepšení
vezmeme průměrnou obtížnost nebo průměrný počet fixací. Kvůli velkému rozptylu
a malému vzorku se nám však existenci tohoto efektu nepodařilo prokázat
statisticky významně.

Další zajímavou otázkou je, zda byl v experimentu nějak závislý průměrný počet
fixací na kontrastu cíle. Na tuto otázku je na první pohled překvapivá odpověď,
že na sobě tyto dvě veličiny závislé nejsou, nebo jen zcela minimálně --
korelace kontrastu cíle a průměrného počtu fixací je $-0.06$. Další hodnoty
jsou vidět na grafu v první příloze (viz obrázek \ref{kontr}), ale krom
extrémních hodnot kontrastu jsou i na grafu přesně, jak napovídá korelace,
všechny průměry přibližně stejné.

Tento závěr by nás však neměl příliš překvapit, dal se očekávat již při návrhu
experimentu. V experimentu totiž účastníci zústali na dané úrovni kontrastu,
dokud se jim na ní nezačalo výrazněji dařit. Jakmile začali dosahovat dobrých
výsledků, byli přesunuti o úroveň níž. Na místě je otázka, jak bychom tedy měli
navrhnout experiment, jehož cílem by bylo prokázat závěry této práce
statisticky signifikantně.

Jedna možnost by byla pokračovat tak, jako v této
práci, a jako ukazatel použít kontrast, příliš nezáleží na tom, zda průměrný, na konci testu, nebo nejnižší dosažený.    
Druhou možností by bylo v prvním a třetím testu obtížnost neměnit


\section{Limitace}

Při návrhu experimentu jsme narazili na několik problémů, které mohly
nepřesnosti do měření, ale jejichž řešení je mimo rozsah této práce. Konkrétně
se jedná o následující obtíže:

\begin{itemize}

\item Vjemy lidského pozorovatele neodpovídají příliš dobře vjemům simulovaného
ideálního pozorovatele. Aby si tyto vjemy odpovídaly, alespoň přibližné, museli
bychom každému pozorovateli změřit jeho vlastní $d'$ mapu. Měření $d'$ mapy ale
i  v té nejminimalističtější variantě, která se používá, trvá nejméně jeden
pracovní den. Druhým důvodem, proč mohou být vjemy rozdílné i v případě, že by
konstanty $d'$ mapy vyšly pozorovateli stejně, jaké byly použity, je, že tato
naměřená $d'$ mapa odpovídá situaci, kdy je scéna se šumem umístěna tak daleko
od pozorovatele, aby ji viděl pod zorným úhlem $15^\circ$. To při velikosti
scény v našem případě odpovídá vzdálenosti pozorovatele a zařízení přibližně
$65 \operatorname{cm}$. V našich experimentech nebyla vzdálenost pozorovatele od
scény hlídána a určitě byla nižší než řečených $65 \operatorname{cm}$
(dodržení této vzdálenosti by odpovídalo situaci, kdy by účastníci drželi iPad
před sebou zhruba na délku natažené paže). 

\item I pokud odhlédneme od nepřesností zmíněných v předchozím bodě a dovolíme
si na chvíli (evidentně scestný) předpoklad, že účastnící měli vlastní $d'$ mapu
konstantní, narazíme na další problém. Okraj oblasti, která byla odkrývána,
byl ztmavován lineárně se snižující se hodnotou $d'$ v použité $d'$ mapě.
Závislost $d'$ na kontrastu ale téměř jistě
není lineární.

\item S tím souvisí ještě jeden problém: Subjekty samozřejmě nemají svou $d'$
mapu konstantní. Tato mapa se tedy nějak skládá s $d'$ mapou, pomocí které
bylo určeno odhalování šumu. V práci jsme toto skládání ignorovali (tedy
předpokládali jsme, že $d'$ na kontrastu závisí lineárně a $d'$ mapa účastníků
je konstantní.) Nabízela by se otázka, proč tedy bylo zatemňování šumu vůbec
prováděno. To se dělo z několika důvodů:

\begin{itemize}

\item Zatemňování šumu zavádí potřebu klikat na místa, která chce pozorovat v
dalším kroku prozkoumat. Nutí tedy pozorovatele, aby tato rozhodnutí dělal
vědomě a nikoli podvědomě, což byl jeden z efektů, které jsme chtěli zkoumat.

\item Celý proces jedné fixace tímto způsobem také trvá mnohem déle (nižší
jednotky vteřin místo nižších desetin vteřiny) a poskytuje nám tedy mnoho času
na update mapy aposteriorních pravděpodobností a výpočet množství informace,
kterou lze získat následující fixací.

\item Takto navržený experiment též umožňuje zjišťovat, které lokace účastník
fixuje bez použití eyetrackeru nebo jiných technologií.

\end{itemize}

\item Zpomalení celého procesu výběru fixace přináší ale i jednu komplikaci. V případě, kdy
účastník provádí jednu fixaci za $300\operatorname{ms}$, nestíhá nad svou
strategií volení fixací přemýšlet. To znamená, že zlepšení strategie je
skutečně způsobeno převážně percepčním učením. V našem experimentu si však
účastníci mohli nad každou další fixací přemýšlet, je tedy pravděpodobné, že
docházelo vedle možného percepčního učení  i ke kognitivnímu učení. Percepční
učení se však v experimentu odehrálo určitě též.  Během experimentu se u
účastníků zjevně podstatně zlepšila jejich senzitivita, ke konci experimentu
byli schopní odhalit cíl s výrazně nižším kontrastem, než na jeho počátku.


\item Vzhledem k tomu, že počet reálných pixelů displeje neodpovídal (a ani
nebyl dělitelný) velikostí scény v pixelech, je možné, že byly efekty jako
například antialiasingem změněny lokální kontrasty scény.

\item Ve výzkumu v oblasti psychofyziky se většinou
pečlivě kontroluje prostředí (například se zatemňuje místnost, v níž se provádí experiment). To jsme v našem výzkumu nedělali.

\end{itemize}

Za zásadní chybu naopak nepovažujeme malý počet účastníků -- kdybychom chtěli
na této práci postavit přesný experiment, nebylo by potřeba zvyšovat počet
účastníků. Jinou, v oblasti psychofyziky často preferovanou cestou, je pokusit
se co nejvíce minimalizovat vliv vnějšího šumu a provádět více měření měření na
jednotlivých účastnících. To bychom mohli uskutečnit například tak, že bychom
vícekrát opakovali první a třetí test. Přístupu, kdy zkoumáme malý počet
účastníků, ale provádíme mnoho co nejpřesnějších měření, se říká small-$N$
design. O jeho výhodách pojednává ve svém článku \citet{SmallN}. Kdybychom se
však rozhodli raději pro Large-$N$ design, mohli bychom většinu ostatních
parametrů experimentu ponechat, ale bylo by potřeba přinejmenším o řád více
účastníků.

V průběhu měření se ukázala jedna chyba v návrhu experimentu, kterou by
bylo při pokračování ve výzkumu vhodné odstranit bez ohledu na to, zda bychom
zvolili small-$N$ či large-$N$ design. 

V aplikaci, pomocí níž byl experiment prováděn, byl naimplementován model obecný
model pozorovatele tak, jak je popsán v článku \citep{Najemnik05}. Tento model
funguje tak, že pozorovatel dostane po fixaci z každé lokace odpověď, která je
číslem náhodně vygenerovaným z normálního rozdělení. Střední hodnota tohoto
rozdělení závisí na tom, zda je v dané lokaci cíl přítomen nebo ne, směrodatná
odchylka závisí na hodnotě $d'$, kterou tato lokace zhledem k zafixované lokaci
má. Takto získané odpovědi potom model použil k update mapy posteriorních
pravděpodobností.

Tento přístup dává smysl v případě, kdy je naším cílem skutečně simulovat
reálného pozorovatele. Reálnému pozorovateli se stejně jako takto simulovanému
občas stane, že vjem z nějaké lokace (typicky dále od místa, které zafixoval,
kde hodnota $d'$ není příliš vysoká) nesprávně vyhodnotí jako pravděpodobnou lokaci
cíle. V takovém případě je samozřejmě racionální následující fixací
zkontrolovat, zda se tam cíl skutečně nachází. Problém nastává ve chvíli, kdy
se takovéhle zavádějící pozorování stane simulovanému pozorovateli, ne však
pozorovateli reálnému. Simulovaný pozorovatel potom svou zpětnou vazbou trvá na
tom, že místo, kde si myslí, že zaznamenal cíl, je nejvhodnější k další fixaci,
často s velkým rozdílem oproti ostatním lokacím. V takové situaci je zpětná
vazba modelu ELM vyloženě matoucí. 

Při opakování experimentu by tedy bylo vhodnější implementovat ELM pozorovatele
tak, aby si myslel, že dostává od modelu odpovědi vybrané z příslušných
distribucí, ve skutečnosti by ale dostával pouze dvě možné konstanty, jednu z
lokací, kde cíl není, a druhou z lokace, kde je. Takto naimplementovaný
pozorovatel by se stále choval dostatečně podobně reálným pozorovatelům, aby
dávalo smysl snažit se naučit lidského pozorovatele jeho strategii, ale zmenšil
by se rozptyl mezi vnitřním šumem pozorovatele a vnitřním šumem modelu.

