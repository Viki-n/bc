\chapter*{Závěr}
\addcontentsline{toc}{chapter}{Závěr}

V~práci jsme představili úlohu zrakového vyhledávání, modely několika pozorovatelů
řešících jednu instanci této úlohy a několik dalších souvisejících konceptů a
teorií. Dále jsme popsali principy percepčního učení.

Poté jsme představili hypotézu, která říkala, že se lidé dokážou naučit řešit
úlohu zrakového vyhledávání lépe, pokud budou dostávat po každém pohledu
zpětnou vazbu. Navrhli jsme experiment, ve kterém jsme se snažili tuto hypotézu
potvrdit či vyvrátit. Ten spočíval v~tom, že jsme nechali účastníky experimentu
opakovaně hledat cíl ve scéně s~vizuálním šumem. Nejprve jsme na několika
takových úkolech změřili výkonnost pozorovatele, potom jsme nechali účastníky
na další sadě úkolů trénovat a nakonec jsme opět změřili jejich výkonnost.
Někteří z~nich přitom dostávali ve fázi tréninku při každé fixaci zvukovou
zpětnou vazbu popisující, jak vhodnou lokaci si uživatel k~fixaci vybral.
K~hodnocení byl použit jeden z~modelů pozorovatelů, konkrétně ELM pozorovatel, který efektivně aproximuje Ideálního Bayesovského pozorovatele.

Vytvořili jsme aplikaci pro operační systém iOS, která gamifikuje úlohu zrakového vyhledávání. Poté jsme s její pomocí provedli navržený experiment. I~přesto, že jsme experiment prováděli pouze na deseti
účastnících (a tedy jsme porovnávali dvě pětičlenné skupiny), podařilo se nám
získat statisticky významný výsledek ($p$-hodnota $0.002$), který říká, že při
opakovaném absolvování úlohy zrakového vyhledávání dochází ke zlepšení výkonu.

Pozitivní vliv zpětné vazby, ač pravděpodobně též velký ($\eta^2_p = 0.17$), se nám však již nejspíše kvůli
malému vzorku a velkému množství vnějšího šumu při experimentech
nepodařilo prokázat. Bylo by zajímavé opakovat experiment s~několika drobnými
změnami, které jsme navrhli při zpracování a diskutování naměřených hodnot.
Výsledky jednotlivých skupin jsou však dostatečně rozdílné na to, abychom věřili, že pokračování
výzkumu tímto směrem má smysl.
